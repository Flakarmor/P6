\chapter{Backlogs} \label{app:product-backlog-s1}
This is a full list of GUI related user stories, including both finished user stories (Release Backlogs) and user stories we have not worked on in this semester (Product Backlog). 

Each user story consist of a title, a priority from 1-5 where 1 is most important, and the story. If the priority there is a hyphen, ´´-''.
Furthermore the user stories is categorised.

The format of the user stories is: \\
``As a (end user), I want to (some goal) so that (some reason)'' \citep{agile-topics}.

In the user stories there is used different kinds of users:
\begin{description}
\item [Developer] People who work on the development of the \giraf, both current and future developers.
\item [Citizen] Children, adults, and elders with an autism spectrum disorder or other developmental disabilities.
\item [Guardian] Legal guardians (e.g. parents) or institutional guardians (e.g. pedagogues).
\item [User] All kind of users of \giraf.
\end{description}

\section{Product Backlog}
This section includes all user stories that are not finished.
\subsection{Structural Changes}
These user stories should have an impact on the structure of the work environment.

\begin{enumerate}
\item \textbf{Replace Redmine with another tool (4)}\\
As a developer I would want to use another tool instead of Redmine so that i could get a more reliable organization tool as the handling of user stories on Redmine is clumsy and is therefore not used by the majority.

\subsection{Server}
These user stories should have an impact on the server or be strongly related to server work.

\item \textbf{Acquire faster servers (3)} \\
As a developer I would like the servers to be faster, so I do not need to wait too long for Jenkins to build my newest release as others could be waiting for the new release.

\item \textbf{Security standards (5)} \\
As a developer I would like the server to follow the security standards set by the Government, so that the server can be moved to a server run by the Government.

\subsection{Jenkins}
These user stories should be related to Jenkins and any work done on Jenkins.

\item \textbf{Monkey testing screen-capture from last failure (2)}\\
As a developer I would like to have screen-captures taken during the monkey test so that i can get feedback about where the failure happened. 

\item \textbf{Add code style check to Jenkins (4)}\\
As a developer I would like to have Jenkins automatically check whether I follow the specified coding style, because it would allow me to focus on developing content rather than spending time on prettying my code.

\item \textbf{Safely disconnecting tablets from testing pool (-)}\\
As a developer I would like to be able to disconnect my tablet from the testing pool safely, without breaking a build on Jenkins.

\subsection{Documentation}
These user stories should be related to the documentation of \giraf.

\item \textbf{Update and extend descriptions of apps and libraries (2)}\\
As a customer and as a developer I would like to have better and more up-to-date descriptions of every app. This should be in some commonly shared place and on Google Play.

\subsection{Stand-Alone apps}
These user stories should be related to the stand-alone concept.

\item \textbf{Stand-Alone: Core package (5)}\\
As a customer I would like the apps to be stand-alone so that i don't have to install any additional app to get one specific app working.

\subsection{Git}
These user stories should be related to the Git-repositories.

\item \textbf{Set up major and minor tags on Git (2)}\\
As a developer I would like to be able to quickly and easily revert to a previous state of the development using tags for every major and minor release.

\subsection{Analysis}
These user stories should be related to analysis of potential new ideas for \giraf.

\item \textbf{Future technology analysis (5)}\\
As semester coordinator I would like to know which future technologies could be useful in the project’s future.

\end{enumerate}

\section{Release Backlog - Sprint 1}
This section includes all user stories from the release from Sprint 1.

\section{Release Backlog - Sprint 2}
This section includes all user stories from the release from Sprint 2.

\section{Release Backlog - Sprint 3}
This section includes all user stories from the release from Sprint 3.

\section{Release Backlog - Sprint 4}
This section includes all user stories from the release from Sprint 4.