\chapter{Recommendations and suggestions for the next semester}
\textit{This chapter is written for the students that take over the project next semester. The section contains recommendations and suggestions our group have for the next semester’s students who will continue the project.}

\section{Issue handling tool}
At the start of the semester we used a issue handling tool from Redmine, but as no one was using it and because it required too much effort to use, we started using Google Docs as a replacement. For the next year we would recommend that a new and better tool in found, so that issue handling becomes easier and faster to use, especially when used for maintaining the backlog as it is critical that a proper backlog is maintained.

\section{Renaming}
This year we renamed several apps to have more meaningful names in order to make it easier to understand what an app was supposed to do. The process used was tedious, and we do not advise any unnecessary renaming. That being said there are some issues with the some of the current names, which we will present here.

\subsection{Categorygame}
The categorygame currently on Google Play is the version using the wrong package-name of train. The correct version has been uploaded to Google Play, and any push to Jenkins on categorygame will upload the alpha-version to the correct version on Google Play, but that version will crash on startup, because the library metadata is referring to train. Fixing this would require making the app compatible with the methods to handle pictograms that was added in the newer versions of dblib.

\subsection{Various other problems}
Ugeplan is named in danish, and could be changed to follow the english names of the other apps.
Some of the names on Git have not been changed, which could and should be finished early.
The Maven repository has not been completely renamed, as only a few libraries were renamed, but it should be done to keep consistency.

\section{Core library}
\label{nextyear_core}
The idea of the Core library is to bypass the launcher, while still having some of the launcher’s features available for other apps to use. Section \ref{Sprint2_SecStandalone} goes into detail about these features. The one major problem of having the core package is to handle data management. A description of these problems and ideas on how to solve them can be found in Section \ref{Sprint3_SecCoreLibrary}.
For the next semester we would recommend that the developers look into whether they want to make the whole project into one package, see Section \ref{Sprint4_PackageSolution}. If the project becomes one package there will be no use for a core package. So it is important to have to discussion about a combined package before trying to solve the core library.

\section{Giraf Package}
\label{nextyear_package}
For the last sprint we were given the task to bundle all the apps together into a single download. Our findings concluded that the only way to do this would be to make all the apps in the project into libraries and then include them in the launcher and by that making the whole project into one app. As this would have taken a lot of time and would have affected many of the other groups, it was decided to postpone it to next year. It is therefore our recommendation that if it is decided to make this change, the change should be made as early as possible.

\section{User Story}
To make user stories easier to write and understand a standard structure of how to formulate them were made. This formulation was “As a \{type of user\}, I want \{some goal\} so that \{some reason\}”.\\
This structure made it clear how the main part of the user story should look, but we later decided that it lacked the constraints and conditions for when the user story was satisfactied. This meant that it in some cases was up to the group that had the user story to decide, when the user story was done. We recommend for the next semester’s  students to improve upon our model by including a standard way of formulating the constraints and conditions for satisfaction.

\section{Backup}
At the end of the semester there were some problems with a disk on the server that hosts the entire project. This meant that there was half a week where the server was down. For the next semester we would recommend having an external backup of critical data, in case something should happen to the server. This backup could be done at the end of each sprint. An idea for a backup solution would be to use Google Drive since there already is 15GB assigned to the Giraf Google account used in the project.(giraf\@lists.aau.dk).

\section{Process}

\subsection{Leadership}
It is our experience that cooperation between students is inherently weak outside of their immediate vicinity and they will often focus on their own projects to the detriment of the greater project if not encouraged otherwise. Thus a structure can help facilitate clear lines of communication and responsibility but such a structure requires overhead to function in the form of meetings and planning.
For this reason, we recommend finding a disciplined group with high standards and a good work ethic for overseeing the process run by the project on all levels.

\subsection{Scrum}
The development method that was used in the project and by our group was scrum. We recommend that scrum is also used for the project next semester, because of its flexibility. We would also advise specifying the process early.

\subsection{Subprojects}
For this semester, the project was divided into three subprojects B\&D, DB, and GUI. At the start of the semester the number of groups was assigned to these subprojects, but over the course of the semester some of the groups moved to the GUI subproject due to the excessive amount of user stories in GUI. We recommend that the next semester keep the subprojects, but have most of the groups in the GUI subproject.
We would also advise not to underestimate the need for further development on the B\&D front or the advantages proper project management brings.

\subsection{Weekly meetings}
Since the development method used in the project was scrum we had weekly meetings where we discussed each subproject’s status and brought up major decisions for the project. However we noticed that many issues which could have been brought up at these meetings were not. We suspect that the reason for this is that in order for anything to be taken up, a person would have to interrupt the meeting to get his question brought up. For this reason we recommend that a suggestion box is made in some form, where everyone can add notes at any time of the week with questions or issues which can then be brought up at the meetings.

\section{Server}
Over the course of the semester the project had some trouble with the group responsible for the servers. There were often very few of the members present at the university. This meant that often when there was problems with the server it took a long time before the problem was fixed. We therefore recommend that for the next semester two groups are responsible for the server or at least make sure that the group responsible for the server is a group that always is at the university at working hours.