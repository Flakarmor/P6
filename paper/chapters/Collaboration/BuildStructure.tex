\section{Build Structure collaboration}\label{Collab_secBuildStructure}
\textit{When we made AAR files a part of the project build structure, a number of groups were involved. This section will describe the collaboration we had with other groups to ensure that the build structure were implemented to the one described in Section \ref{Sprint2_buildstructure}}

In our work with dependencies and making a standalone app, we found that the build structure could be improved by using AAR files. We then shared the idea with our subproject for its use to reform the build structure. It was agreed to implement AAR files and tasks were assigned between the relevant groups. From this point onwards, we acted as consultants for the other groups as it was implemented.

As part of the consultation we found the dependencies for the apps and libraries as described in Section \ref{Sprint2_SecDependencies}. This was used to update the Gradle build files used to correct the libraries. The Gradle build files were then updated to use the Maven plugin by ourselves and the Jenkins group, which would allows projects to easily implement libraries, by simply asking for the libraries from Maven. The plugin downloads the implemented library on build time. This means that updated libraries can be used as a project without having to change any of the project's code. Maven also allows for backwards compatibility by using version numbers, which allows the individual projects to use the version of the libraries they need. Furthermore our work with the dependencies helped as consultation with the Git group when the recursive build structure had to be removed.
