\section{Summary}
\textit{This section is a short description of what we did in the 3rd sprint, and which user stories that followed over to the next sprint.}\\

The main goal of this sprint was to handle the renaming of most apps and some of the libraries, as well as ensuring that the various guides used in the project was available on Redmine in a readable format.

\textbf{Renaming of app and library-package-names:}
As the naming of apps and libraries were inconsistent, it was decided that they should be made consistent. We were the group responsible for ensuring this consistency overall, and we had to rename the package-names in specific, so the apps could be uploaded with the new package-names to Google Play. To accomplish this we made a detailed guide on how to do the renaming. The renaming was at the end of the sprint finished for the most part but some names still needed to be changed on the Git-repositories.

\textbf{Guide consistency:}
We reformatted and to a minor extent expanded or rewrote the guides used in the project, so they were more readable. This included taking several guides in pdf-format and writing them on the Redmine Wiki.\\

\textbf{Core library:}
We presented some solutions to the problems at hand, and why we would not implement them ourselves, because it would create conflicts with the work done in the DB subproject. The solutions are further discussed in Section \ref{nextyear_core}.

We completed most of the user stories assigned to us in the 3rd sprint, but some of the renaming was left for the 4th sprint so it would not interfere with the sprint end of the 3rd sprint. Providing code documentation for AAR files was not possible, because it was not supported by Android Studio. The work with the backlog is covered in Section \ref{Roles_SecProductOwner}. We also continued our work with Google Play, Google Analytics, and as Product owners.