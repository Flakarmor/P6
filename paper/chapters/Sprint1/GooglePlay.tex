\subsection{Google Play}
Google Play is a digital distribution platform created by Google. It is used as an appstore for Android devices as well as for music and ebook distribution. Because the goal of the project is to make apps for android devices, we are using the Google Play store to distribute the different apps. \citep{GooglePlay}

As it was our responsibility to maintain and control Google Play for this semester, we started by familiarizing ourselves with the different features provided by Google Play. We found that Google Play provides a lot of statistics for all the apps, like how many times the apps have been installed and how many are still using them. Google Play also allowed os to change which apps should be shown on the store and to change the descriptions of the apps. One thing we noticed when we looked at some of the descriptions was that a lot of the descriptions were very short and did not provide enough information about the apps, so the user would have trouble knowing the functionalities that the apps provide. Because of this, we decided to update the descriptions, make an English version of the descriptions, and to update some of the pictures showing the apps to the user.

One data that Google Play provided ,that was very important, was the statistic about with version of android the users where using. because from that we found that the apps has to use the Android API 15, because 22\% of the users where using android version 4.0.3 - 4.0.4, which only supports API 15 and below \citep{API15}. 

\subsubsection{Alpha/Beta}
At the conclusion of the last semester's work on the Giraf project, all the apps from the project were uploaded to Google Play. This meant that the apps became available for the users, and that data about crashes could be sent to Google Play and Google Analytics. The initial release was plagued by crash issues and other errors. Some of these issues were attributed to certain applications crashing when trying to use other apps.

To better avoid this for the current semester, it was decided to add an alpha and beta release version to Google Play. The alpha and beta version fulfill the same role of ensuring the build is stable before release, but they are updated very differently. The alpha version will be updated whenever a new build is uploaded onto the Jenkins server and it passes the automatic tests to ensure it doesn't break the build. If it passes all of that, this new build will replace the alpha version. The beta will be updated at each sprint end to the newest alpha build. There weren't any plan for when to update the release, but it follows that stable beta versions can be released.

For the work in this implementation of alpha and beta versions, we found a plug-in for Jenkins that could be used to make the uploading of alpha and beta versions automatic, as well as providing the information the server needs to upload to Google Play. The plug-in and information was then passed on to the group responsible for the Jenkins server, so they could implement it.