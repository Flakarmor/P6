\subsection{Google Analytics}\label{Sprint1_SubsecGoogleAnalytics}
Google Analytics is a service created by Google. Google Analytics main function is to track visitors, to show the developer how people are using the product. Mobile App Analytics is a branch of Google Analytics, which focus on mobile apps. Mobile App Analytics has Google Play integration which allows for visitors and traffic sources tracking. This traffic can then be used to see who is installing the apps on which platforms. Mobile App Analytics's API provides for tracking events, crashes, exceptions, and user patterns. \citep{GoogleAnalyticsWhat}

Mobile App Analytics and Google Play have a lot of the same functionality for tracking who installed what on which platforms. However Mobile App Analytics has an advantages when it comes to crash and exception rapports, because the API send a crash report directly to Mobile App Analytics, without the user having to do anything, whereas Google Play does not include crashes that are not reported by users. For this reason Mobile App Analytics were used as the main crash report service.

For crash reporting there were three tasks at hand. The first was to automatically forward the crash reports from an app to the groups that worked on that app. The second and third task were to provide a guide on how to implement the API in the applications and to provide a guide for the next years Google Analytics group on how to autoforward crash reports. 

\subsubsection{Autoforward crash reports}
Mobile App Analytics has a built-in email feature, which allows the user to automatically email a crash report daily, weekly, monthly, or quarterly. It is not intuitive how to autoforward crash reports because the user has to first make a report that is able to fit to the time interval the user want. Therefore, a guide was made which can be found in Appendix \ref{ChapCrashReport}.

\subsubsection{Mobile App Analytics API}
In order to get crash reports for an app sent to Google Analytics, the Crash API for Mobile App Analytics has to be implemented in the app. The API makes it possible to make a Tracker which can be used to report information to Mobile App Analytics, when an exception is caught. A great feature about the tracker is that it can be set to report uncaught exceptions. A guide how to implement Mobile App Analytics API in an app can be found in Appendix \ref{Appendix_GoogleAnalyticsGuide}.
\newpage