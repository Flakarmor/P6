\chapter{Introduction} \label{ChapIntroduction}
GIRAF is a series of applications(apps) for Android, intended to help citizens with autism in their everyday life. Over the past four years GIRAF have been developed by students at Aalborg University, with a new group of students taking the mantle every year. As a result of this, it is hard to get an overview of the project in its entirety. We here give an overview of the GIRAF project when we started working on it.

\section{Status of GIRAF}
Overall GIRAF is in a somewhat functional state with several small to severe problems regarding stability, reliability, and usability.
GIRAF is published in Google's Play Store with the individual apps being version 1.0 while the launcher-app is in version 2.2. The version-numbering is not entirely correct, as some of the apps does not function correctly yet and as such they should not have been published as official versions but rather as alpha or beta versions. The majority of the apps does function correctly, but only a few of them are ready for public release.
Google Analytics was incorporated in roughly a third of the apps, with between three and six months of data on usage and crashes being available. The data has been gathered since the newest version of the apps were published.

\section{Subprojects and Responsibility}
This year the project was divided into three subprojects with the various groups belonging to one of these, as opposed to last year where every group was responsible for a certain app. The three subprojects are: Database (DB), Graphical User Interface(GUI), and Build and Deployment(B\&D).

\subsection{Database}
The groups working with the database are responsible for setting up the database, and ensuring that the data is accessible.

\subsection{Graphical User Interface}
GUI groups are mainly responsible for handling changes in the different apps, as well as bug fixing anything related to the apps.

\subsection{Build and Deployment}
B\&D is responsible for most of the internal project matters, such as servers, automatic build tools, publication, and version control.