\section{Standalone} \label{Sprint2_SecStandalone}
\textit{The customer showed some interest in having some of the apps available as standalone. This section focuses on our work in this area covering which features would be needed to make the existing apps standalone, and our decision to make a core library.}

\subsection{Motivation}
Making the apps standalone would increase the usability of the individual apps, as the user would no longer need to install several different apps just to ensure that they can use the app they want. As seen in Section \ref{Sprint2_SecDependencies} all apps in Giraf are reliant on both the launcher and ‘pictosearch’. This dependency was not only annoying for the user but was also causing trouble during the automatic tests, as one of the tests include installing the apps and testing them pseudo-randomly, but the emulation used for this test could not at the time install more than one app at a time, which resulted in this test always failing on apps reliant on the launcher.

\subsection{Features}
In order to make the individual existing apps standalone, we first looked at each app and found out which tasks the apps had to be able to perform when working standalone. In the following section we describe which features that would enable an app to be standalone.

\subsubsection{Sequence}
Sequence is an app that allow the user to make sequences of pictograms for use in communication. In order to use the app, it needs access to a database with pictograms, and if the database has not been made by the launcher, sequence has to be able to make a database and download the pictograms to the database itself. In order to allow the user to use personal pictograms, the app also need a login feature, so that the user can access these pictograms as well.

\subsubsection{Other apps}
Like sequence many of the other apps that need to be standalone, need access to pictograms as well as personal pictograms. Therefore, we will not go into details with these apps. 

\subsubsection{Overall necessities}
So from this we have arrived to the following list of features the apps would need to work as standalone apps. 

\textbf{Login}: This features will allow the user to access personal pictograms and save pictograms to a personal account.\\
\textbf{Make Localdb}: This feature has to check if a local database of pictogram has already been made by another app. If no local database has been found the app will then need to make it itself.\\
\textbf{Download pictogram}: This feature has to download pictograms for the local database from the server. It also has to get personal pictograms for the user, if the user is logged in to a personal account.

\subsection{Standalone app or library}
As all the apps need the features just mentioned, it would make sense to place them in a common place, either as an app or as a library. That way multiple apps could use the same code without having to rewrite the code in the individual apps. Additionally any new common features needed could be added to this app or library. In general this will be referred to as the core app or core library.
In Table \ref{coreapp_corelib_comp} an overview is shown of the positive and negative aspects (pros and cons) of each solution. The two solutions were evaluated on how easy they would be for the user. This meant that the app-solution would be a bad solution because it would require the user to install another app in addition to the standalone app they wanted to use. The bigger size of the apps in the library-solution is regarded as a minor problem as currently the storage space for the pictograms used far outweighs the storage used by the installed apps and Giraf in its entirety does not use a lot of space to start with.

\begin{table}[H]
	\centering
	\begin{tabularx}{\textwidth}{>{\raggedright}Xp{\textwidth/2}p{\textwidth/2}}
		\hline
		Core standalone App & Core standalone Library \\ \noalign{\vskip 2mm}
		\hline \textbf{Pros} & \textbf{Pros}\\ \noalign{\vskip 2mm}
		
		\hline Smaller in size when more than 1 standalone app is installed & True standalone (no other apps needed)\\ \noalign{\vskip 2mm}
		
		\hline \textbf{Cons} & \textbf{Cons} \\ \noalign{\vskip 2mm}
		
		\hline Another app needs to be installed & Bigger app size\\ \noalign{\vskip 2mm}
		
		\hline \textbf{Other notes} & \textbf{Other notes}\\ \noalign{\vskip 2mm}
		
		\hline Fuse core app and pictosearch (only one service app) & Make pictosearch a part of the core library
		(no service app)\\ \noalign{\vskip 2mm}
		\hline
	\end{tabularx}
	\label{coreapp_corelib_comp}
	\caption{Pros and cons of a core standalone app and a core standalone library}
\end{table}

%\subsection{User story for sprint 3}
%We didn't finish making a standalone application for the second sprint. But with the knowledge we gained from working on the user story, we learned enough to fundamentally change the standalone user story from being for individual apps to a possible core library solution.