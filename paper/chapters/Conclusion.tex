\chapter{Conclusion} \label{ChapConclusion}
The conclusion have two aspects, one for the overall Giraf-project, and one for our own work.

Overall the Giraf-project is in a much more refined state then it was, when we started working on the project.
The apps now have a more consistent design in order to give the user the feeling that they are using a combined system and not ten different apps. The new design has not been implemented in all parts of the project.
The database is now able to synchronize between two tablets running the Giraf-system. This is still in an early state and is therefore not ready to be released to the customers. The functionalities are also still limited to text strings and not pictograms which is one of the end goals for the synchronization to be capable of. Furthermore there is no security incorporated in the synchronization, which would cause severe issues if left unhandled in a released state.

As for our own work we have helped the overall project by improving the build structure of all apps. This was accomplished by changing the libraries from being build recursively on compile time, to being built once and then stored in a Maven repository where the compiled version of the library can then be pulled from. This resulted in a reduction in the build time when new work on an app was pushed to Jenkins. This change among others have helped to reduce the build time of apps from 25-30 minutes when we started the project, to 5-10 minutes at the end of the project. 
An internal renaming of the apps was done to achieve more consistency in the names. Our part of this renaming forced us to re-upload most apps on Google Play. This was done successfully but some of the apps did not work due to some wrong dependencies, which was not fixed. This means that all but one app has been renamed.

We have also done a lot of work as product owners for our sub-project Build and Deployment. As such we helped organise the work for the other groups in the sub-project to some success. Our work also resulted in a combined product backlog that has user stories for all three sub-project that have not been completed yet, and in addition to this a release backlog with all the user stories that have been completed during our time working on the project separated into sprints. This will give the students next year a place to start their work without needing to search through everything that have been done this year, just to find the unfinished user stories.
Our work as a whole was a success and a positive contribution to the Giraf-project. Working with the scrum method fitted the project well, but working in three sub-projects caused some problems. Information was not always shared probably between the sub-projects, which gave some issues along the way. On the other hand it was nice to have specialised groups that could focus on harder to solve problems which could therefore be solved for the most part.
