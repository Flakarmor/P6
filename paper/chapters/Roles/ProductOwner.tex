\section{Product Owner} \label{Roles_SecProductOwner}
\textit{This section will discuss our groups function as the product owners for the semester.}

%Motivation
Our group accepted the role as product owners for the B\& D subproject group in the beginning of the second sprint as more oversight was required than initially expected. Our tasks were to manage User stories and manage the transition between sprints.

In the same move as the product owners were made, a hierarchy of who answers to whom was also designated. As B\& D, we would answer to the GUI and DB subprojects, effectively making them our primary customers. 

%Sprint Planning
As product owners, we began to chair the sprint planning for our subproject. When sprint planning we met and went through every user story in the backlog, making sure our relevant customers would prioritize them. This would be set by representatives of the GUI subproject and to a smaller extent by the DB subproject.

After all the user stories were prioritized, then the B\& D subproject would by themselves divide the most important user stories in between them. Given our rather small focus area split between Git, Jenkins, the server, and Google services, most user stories were bound to these roles.

When we were done giving out user stories, each group left to begin work or estimate, depending on their work ethos, after which they could contact us to renegotiate their user stories.

%Sprint - day to day
While the sprint was in operation, there was little to do as product owners. We were expected to keep the backlog updated with new user stories received either from other sub-projects or from our own. Other groups were expected to come to us with user stories.

Otherwise, we held no special distinction in the weekly group meetings or bi-weekly scrum meetings with the subproject.

%Sprint End
As we began to take over product owner responsibilities, we also began to chair the sprint ends for B\& D. Once all were assembled for the meeting, each group in the subproject would present their work they had done for the sprint. Our Scrum process specified that presentation slides were outlawed for these meetings.

\subsection{Release Backlog} \label{Roles_SecReleaseBacklog}

%Introduction + motivation?
In the third sprint we were tasked with making a release backlog for the Build \& Deploy subproject as a part of the overall Scrum process and to be able to provide a better overview of what has been made during the semester. Being the product owners of Build \& Deploy, we were in a unique position to solve this user story.

In itself, making a release backlog is an almost trivial task. However, since the product owner role wasn't given till the second sprint, there had been done nothing to store any completed user stories and till this sprint, none had considered to start for this subproject.\\
Therefore, the challenge in making a proper release backlog lies in some data being lost or hard to retrieve without exhaustingly going through all the reports of the year or interviewing each group. This possibly leaves the current release backlog partially incomplete.

%What is a release backlog
A release backlog is a subset of the product backlog used for Scrum. Where the product backlog is a list of features wanted for the product, the release backlog contains the user stories which have been completed for a given sprint.

%The release backlog
\begin{table}
	\centering
	\begin{tabular}{ll}
		\textbf{User Stories} & \textbf{Completed by}\\ \hline \noalign{\vskip 2mm}
		Setting up server & Group 2\\ \hline
		Update to Android 1.0.x & Group 8\\ \hline
		Setting up Git & Group 8\\ \hline
		Google Analytics Setup & Group 5\\ \hline
		Setup of Google Play & Group 5\\ \hline
		Central Documentation & Group 5, 9\\ \hline
		Continuous build and integration & Group 9\\ \hline
		Wiki entry on process & Group 9\\ \hline
	\end{tabular}
	\caption{Release Backlog for sprint 1.}
	\label{Roles_ReleaseBacklogSprint1_table}
\end{table}

\begin{table}
	\centering
	\begin{tabular}{ll}
		\textbf{User Stories} & \textbf{Completed by}\\ \hline \noalign{\vskip 2mm}
		Binaries instead of submodules & Group 5, 8, 9\\ \hline
		Snapshots of database/Backups & Group 2\\ \hline
		SMTP setup & Group 2\\ \hline
		Logging & Group 2\\ \hline
		Gitlab for the server/GOG & Group 2\\ \hline
		Release new APK for Sprint 1 & Group 5\\ \hline
		Auto Upload Beta and Alpha release & Group 9\\ \hline
		Monkey test & Group 9\\ \hline
		Test case installation & Group 9\\ \hline
		Faster build and test & Group 9\\ \hline
		Code coverage & Group 9\\ \hline
		Dependency & Group 9\\ \hline
	\end{tabular}
	\caption{Backlog for sprint 2.}
	\label{Roles_ReleaseBacklogSprint2_table}
\end{table}

\begin{table}
	\centering
	\begin{tabular}{ll}
		\textbf{User Stories} & \textbf{Completed by}\\ \hline \noalign{\vskip 2mm}
		Logging & Group 2\\ \hline
		Symmetric DS & Group 2\\ \hline
		Snapshots of database/Backups & Group 2\\ \hline
		Share the knowledge from Guides & Group 5\\ \hline
		Add a guide on how to do UI test & Group 9\\ \hline
		Make guidelines for Continuous Integration & Group 9\\ \hline
		Specify the Scrum process used & Group 9\\ \hline
		Monkey test & Group 9\\ \hline
		UI test on Jenkins & Group 9\\ \hline
	\end{tabular}
	\caption{Backlog for sprint 3.}
	\label{Roles_ReleaseBacklogSprint3_table}
\end{table}