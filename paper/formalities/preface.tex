\chapter*{Preface}

\subsubsection{Reading guide}
\textit{This section is a reading guide which contains a glossary which is containing explanations to words and terms that are not apparent from the context they are in.}

In addition, it should also be noted that each chapter and section starts with a short introductory description in italics.

\subsubsection{Word list}

\begin{description}
  \item[Our, We:] Refers to our own group.
  
  \item[Customer:] Refers to the external customers from the institutions the project is working with.
  
  \item[Project:] Refers to the Giraf project as a whole.
  
  \item[Subproject:] Refers to one of the three parts of the project. The project was split into Build and Deploy (B\&D), Database (DB), and Graphical interface (GUI). If no specific subproject is stated it refers to the B\&D subproject.
  
  \item[App:] Picked in favor of the longer ‘application’.
  
  \item[Launcher:] Refers to the launcher, which is the main app in the Giraf-system from which all the other apps are executed.
\end{description}

\subsubsection{Glossary}

\begin{description}
  \item[Activity:] An activity is a class that takes care of creating a window, which is used to interact with the user through UI elements. An activity can be created or destroyed from other activities. Activities can work in the background or be set to pause when another activity comes to the foreground, to insure that not all activities use memory, when not in the foreground. 
  
  \item[Pictogram:] A pictogram is a drawing that represent a physical object with some short explaining text, which can then be used by a person with autism to communicate.
  
  \item[Package-name:] A package-name is a text-string used by Google Play to uniquely identify an app. We use the newest package names to refer to the apps.
\end{description}